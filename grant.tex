\documentclass[12pt,conference,onecolumn]{IEEEtran}

\usepackage[hidelinks]{hyperref}

\title{Rock paper scissors robot arm}
\author{%
\IEEEauthorblockN{Gage Grant}\IEEEauthorblockA{Science \& Engineering\\Manalapan High School\\Englishtown, NJ\\\href{mailto:426ggrant@frhsd.com}{426ggrant@frhsd.com}}\and
\IEEEauthorblockN{Jason Donovan Katz}\IEEEauthorblockA{Science \& Engineering\\Manalapan High School\\Englishtown, NJ\\\href{mailto:426jkatz@frhsd.com}{426jkatz@frhsd.com}}
}
\date{June 16, 2026}

\newcommand{\keywords}{rock, paper, scissors, robot arm, OpenCV, machine vision, cues}

\usepackage{hyperref}
\makeatletter
\AtBeginDocument{
\hypersetup{%
pdftitle={\@title},
pdfauthor={Gage Grant and Jason Donovan Katz},
pdfkeywords={\keywords}}}
\makeatother

\begin{document}
\maketitle 

\begin{abstract}
Rock Paper Scissors is one of the most recognizable, easy, and popular hand games out there. And while it is sometimes fun to play as a last resort in an incredibly boring situation, winning roughly 50\% of the time isn't good enough for us. We want to not only win, but crush the competition. Humans are terrible at being random, and we want to make a robotic hand to exploit this human weakness.

Our main goal for this semester project is to make a robotic hand that will beat humans at rock paper scissors. Using OpenCV and some basic electronics parts, we want to have a basic robotic hand that can recognize a pattern from a human in rock paper scissors, and exploit it to crush them, in the game.
\end{abstract}

\begin{IEEEkeywords}
\keywords
\end{IEEEkeywords}

\end{document}
